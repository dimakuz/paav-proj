\documentclass[12pt,a4paper]{article}
\usepackage{fullpage}
\usepackage[top=2cm, bottom=2cm, left=2.5cm, right=2.5cm]{geometry}
\usepackage{amsmath,amsthm,amsfonts,amssymb,amscd}
\usepackage{enumerate}
\usepackage{xcolor}
\usepackage{listings}
\usepackage{graphicx}
\usepackage{caption}
\usepackage{subcaption}

\lstset{
frame = single, 
breaklines=true,
framexleftmargin=1pt}

\title{Program Analysis and Verification:\\Final Project}
\author{Ilya Shervin\\ 308640218 
        \and
        Dmitry Kuznetsov\\ 322081183}


\newcommand{\trans}[1]{[\![\texttt{#1}]\!]}
\newcommand{\atrans}[1]{\trans{#1}^\sharp}
\lstset{basicstyle=\small}
\newcommand{\example}[3]{
\input{#1/examples/example#2.tex}
\begin{figure}[h!]
\centering
\begin{subfigure}{.5\textwidth}
  \centering
  \includegraphics[height=#3\textheight]{./#1/examples/example#2-output/cfg.png}
  \caption{Control flow graph}
\end{subfigure}%
\begin{subfigure}{.5\textwidth}
  \centering
  \lstset{basicstyle=\small}
  \lstinputlisting{#1/examples/example#2.code}
  \caption{Code}
\end{subfigure}
\caption*{\texttt{pipenv run analyze --type #1 docs/#1/examples/example#2.code}}
\end{figure}
}

\newcommand{\parfBasis}[1]{(#1_{even}\leftrightarrow\neg #1_{odd})}

\begin{document}
\maketitle
	\section*{Intro}
	This document accompanies the code of the final project in Program Analysis and Verification class. Following is a brief rundown of code structure and components.
	
	All of the analysis tasks are implemented with Python 3 (3.7+) with the environment and dependencies managed by Pipenv\footnote{https://docs.pipenv.org} (see docs for elaborate usage examples).
	
	Our code relies on the following external software packages:
	\begin{itemize}
		\item SLY\footnote{https://github.com/dabeaz/sly} - a Python implementation of lex/yacc analogues lexers and parsers.
		\item pySMT\footnote{https://github.com/pysmt/pysmt} - a front-end for SAT and SMT solvers
		\item z3-solver\footnote{http://z3prover.github.io} - a back-end for PySMT based on Z3
		\item graphviz\footnote{https://github.com/xflr6/graphviz} - a library for visualization of control flow graphs
		\item sympy\footnote{https://www.sympy.org/} - a library implementing symbolic algebra, used in shape analysis node length computations
		\item pytest\footnote{https://docs.pytest.org} - a test harness
	\end{itemize}
The general method of operation for all of the analysis implementations is as following:
\begin{enumerate}
	\item Parse the input file according to the language definition of the program (parity/sum/shape).
	\item Construct the control flow graph of the program.
	\item Initialize all the nodes in control flow graph to the initial abstract state (depending on the abstract domain of specific analysis).
	\item Perform chaotic iteration until no more updates happen.
	\item Iterate over assert statements and verify their validity against the abstract state of the graph node.
\end{enumerate}
Finally, the results are either printed to the screen as DOT graphs or written to disk as PNG files.

\section*{Installing and Running}
Development of the analysis was done on Linux (any recent Fedora/Ubuntu). To set up a Python environment, run \texttt{pipenv install} in the root directory of the project. This will initialize a new virtual environment and pull required packages. Additionally several system packages are required for execution:
\begin{itemize}
	\item \texttt{graphviz}
	\item \texttt{z3solver}
\end{itemize}

Invocation of analysis is done through \texttt{pipenv run analyze}:
\lstinputlisting{analyze.usage}

The project directory layout is as following:
\begin{itemize}
\item \texttt{analyzeframework/} - Code for basic components used in all analyses (e.g. chaotic iteration, control flow graphs, visualization).
\item \texttt{analyzenumeric/} - Code specific for parity and summation analyses.
\item \texttt{analyzeshape/} - Code specific for shape analysis.
\item \texttt{examples/} - Code examples for all three kinds of included analysis.
\item \texttt{docs/} - Code for building this file.
\item \texttt{analyze.py} - Main entry-point of our program.
\item \texttt{tests/} - Automatic tests for our analyses code.
\end{itemize}
\newpage
\part*{Parity analysis}
\section*{Overview}
Our implementation of shape analysis is based on the \href{https://www.cs.tau.ac.il/~msagiv/toplas02.pdf}{paper} by Sagiv, Reps and Wilhelm "Parametric Shape Analysis via 3-Valued Logic", with modifications to allow for symbolic length analysis between nodes in a linked list. We try to capture the original aspects of execution that are discussed in the paper, along with the addition of abstract length calculations between nodes in a list.
\begin{itemize}
	\item Storing instrumentation predicates for a node in the heap (concrete or abstract) as defined in the paper - such as reachability predicate. The predicates are evaluated in 3-valued logic.
	\item Storing a symbolic size for an abstract node.
	\item Storing an array of possible heap structures according to the current execution state.
\end{itemize}
With such information, we are able to go over all possible structures and verify that our \texttt{assert} statements hold in each of them.
\section*{Abstract Domain}
Our abstract state is defined as a 2-tuple of the from $(C,D)$. Where:
\begin{itemize}
	\item $C: Symbols \rightarrow \mathbb{Z}\cup \{\bot, \top\}$ - a mapping that tracks constant values of variables.
	\item $D: (Symbols \times Symbols )\rightarrow \mathbb{Z}\cup \{\bot, \top\}$ - a matrix that tracks difference between values of variables.
	
	For example, if $x$ is known to be greater than $y$ by 2, then $D[x,y] = 2$ and $D[y,x] = -2$.

\end{itemize}
Our abstract domain is then all possible assignments to the 2-tuple.

We'll define a $\sqcup$ operation between an integers and $\{\bot, \top\}$:
\begin{itemize}
	\item $K \sqcup \bot = K$
	\item $K \sqcup \top = \top$
	\item $K \sqcup K = K$
	\item $K \sqcup K^\prime = \top$
	\item $\bot \sqcup \top = \top$
\end{itemize}
We'll also define the above $\sqcup$ commutative, meaning that $a\sqcup b \equiv b \sqcup a$.

The join operation is defined as following: given $\alpha_1=(C_1, D_1)$ and $\alpha_2=(C_2,D_2)$ abstract states:
\begin{align*}
&\alpha_1\sqcup\alpha_2=(C_1\sqcup C_2, D_1 \sqcup D_2) \\
\forall s \in Symbols&:  \\
&(C_1\sqcup C_2)[s] = \left.
	\begin{cases}
		C_1[s]\sqcup C_2[s], & s \in Symbols
	\end{cases}
\right\} \\
&(D_1\sqcup D_2)[s_1, s_2] = \left.
	\begin{cases}
		D_1[s_1, s_2]\sqcup D_2[s_1, s_2], & s \in Symbols
	\end{cases}
\right\}
\end{align*}
\section*{Abstract Transformers}
The transformers used in the shape analysis are implemented exactly as described in the paper and in the class presentation (where they are refered as update formulae). There are update formulae for all the predicates $x$, $r_x$ (for every $x\in Symbols$), $n$, $c$ and $is$, for each statement.

The only addition to the original algorithm is that in \texttt{x := new} statement, we set $size(v) = 1$ where $v$ is the new individual that \texttt{x} points to ($x(v)=1$).

\newpage
\subsection*{Test programs}
\subsubsection*{Example 1 - Known parity}
\example{parity}{1}
\subsubsection*{Example 2 - Tracing similar parity}
\example{parity}{2}
\newpage
\clearpage
\part*{Summation analysis}
\section*{Overview}
Our implementation of shape analysis is based on the \href{https://www.cs.tau.ac.il/~msagiv/toplas02.pdf}{paper} by Sagiv, Reps and Wilhelm "Parametric Shape Analysis via 3-Valued Logic", with modifications to allow for symbolic length analysis between nodes in a linked list. We try to capture the original aspects of execution that are discussed in the paper, along with the addition of abstract length calculations between nodes in a list.
\begin{itemize}
	\item Storing instrumentation predicates for a node in the heap (concrete or abstract) as defined in the paper - such as reachability predicate. The predicates are evaluated in 3-valued logic.
	\item Storing a symbolic size for an abstract node.
	\item Storing an array of possible heap structures according to the current execution state.
\end{itemize}
With such information, we are able to go over all possible structures and verify that our \texttt{assert} statements hold in each of them.
\section*{Abstract Domain}
Our abstract state is defined as a 2-tuple of the from $(C,D)$. Where:
\begin{itemize}
	\item $C: Symbols \rightarrow \mathbb{Z}\cup \{\bot, \top\}$ - a mapping that tracks constant values of variables.
	\item $D: (Symbols \times Symbols )\rightarrow \mathbb{Z}\cup \{\bot, \top\}$ - a matrix that tracks difference between values of variables.
	
	For example, if $x$ is known to be greater than $y$ by 2, then $D[x,y] = 2$ and $D[y,x] = -2$.

\end{itemize}
Our abstract domain is then all possible assignments to the 2-tuple.

We'll define a $\sqcup$ operation between an integers and $\{\bot, \top\}$:
\begin{itemize}
	\item $K \sqcup \bot = K$
	\item $K \sqcup \top = \top$
	\item $K \sqcup K = K$
	\item $K \sqcup K^\prime = \top$
	\item $\bot \sqcup \top = \top$
\end{itemize}
We'll also define the above $\sqcup$ commutative, meaning that $a\sqcup b \equiv b \sqcup a$.

The join operation is defined as following: given $\alpha_1=(C_1, D_1)$ and $\alpha_2=(C_2,D_2)$ abstract states:
\begin{align*}
&\alpha_1\sqcup\alpha_2=(C_1\sqcup C_2, D_1 \sqcup D_2) \\
\forall s \in Symbols&:  \\
&(C_1\sqcup C_2)[s] = \left.
	\begin{cases}
		C_1[s]\sqcup C_2[s], & s \in Symbols
	\end{cases}
\right\} \\
&(D_1\sqcup D_2)[s_1, s_2] = \left.
	\begin{cases}
		D_1[s_1, s_2]\sqcup D_2[s_1, s_2], & s \in Symbols
	\end{cases}
\right\}
\end{align*}
\section*{Abstract Transformers}
The transformers used in the shape analysis are implemented exactly as described in the paper and in the class presentation (where they are refered as update formulae). There are update formulae for all the predicates $x$, $r_x$ (for every $x\in Symbols$), $n$, $c$ and $is$, for each statement.

The only addition to the original algorithm is that in \texttt{x := new} statement, we set $size(v) = 1$ where $v$ is the new individual that \texttt{x} points to ($x(v)=1$).

\newpage
\subsection*{Test programs}
\subsubsection*{Example 1 - Known parity}
\example{parity}{1}
\subsubsection*{Example 2 - Tracing similar parity}
\example{parity}{2}
\end{document}
