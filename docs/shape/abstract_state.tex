\section*{Abstract Domain}
Our abstract state is defined as a set of possible heap structures. Each structure $S = \langle U^S,\tau^S,s^S\rangle$ consists of the following:
\begin{itemize}
	\item $U^S$ - a set of individuals (heap nodes) that can be concrete or abstract.
	\item $\tau^S$ - a set of $k$-ary predicates $\tau^S$ from a $k$-tuple of individuals to a value in 3-valued logic ($\{0,1,1/2\}$).
	\item $s\in U^S\to Expr$ - a size function that maps an individual to a symbolic algebraic expression.

\end{itemize}
Our abstract domain is all the possible sets of heap structures.

The logical operators are naturally extended to support 3-valued logic in a manner described in class and in the paper:
\begin{itemize}
	\item $1 \lor 1/2 = 1$
	\item $0 \lor 1/2 = 1/2$
	\item $1/2 \lor 1/2 = 1/2$
	\item $1 \land 1/2 = 1/2$
	\item $0 \land 1/2 = 0$
	\item $1/2 \land 1/2 = 1/2$
	\item $\neg 1/2 = 1/2$
\end{itemize}

The set of $k$-ary predicates that we keep for each structure is as follows:
\begin{itemize}
	\item Unary predicate $x\in U^S\to \{0,1,1/2\}$ for each variable $x$ in the program, meaning the variable points to the individual.
	\item Unary predicate $r_x\in U^S\to \{0,1,1/2\}$ for each variable $x$ in the program, meaning the individual is reached from the variable.
	\item Unary predicate $c\in U^S\to \{0,1,1/2\}$, meaning that the individual is on a directed cycle.
	\item Unary predicate $is\in U^S\to \{0,1,1/2\}$, meaning that the individual is pointed to by two or more nodes.
	\item Unary predicate $sm\in U^S\to \{0,1,1/2\}$, meaning that the individual is a summary (abstract) node.
	\item Binary predicate $n\in U^S\times U^S\to \{0,1,1/2\}$ meaning that there is an edge pointing from one individual to another.
\end{itemize}

We define $Expr$ as the set of  

In non \texttt{assume} statements join operation is simply the set union of the possible structures. In \texttt{assume} statements (which are loop heads admittedly), we perform set union but may add an extra structure that introduces an arbitrary size $a_S$ that participates in size expressions. Join operation is described in more detail in the next section. 

When verifying \texttt{assert PRED} statements, our analysis computes satisfiability of the following formula: $state_\alpha\land\neg pred$. We'll detail how $state_\alpha$ is obtained from $\alpha = (C,D)$:
\begin{align*}
state_\alpha & = const_\alpha \land diff_\alpha \\
const_\alpha &= \bigwedge\{s = C[s] \mid s \in Symbols \wedge C[s] \in \mathbb{Z}\}
\end{align*}
In this model, atoms are variables of the program. $const_\alpha$ is the set of clauses that bind variable to concrete integer values.
\begin{equation*}
diff_\alpha = \bigwedge\{ x - y = D[x,y] \mid x,y\in Symbols \wedge D[x,y] \in \mathbb{Z}\}
\end{equation*}
$diff\alpha$ expresses the known differences between variables.

With the above formula, and the formula derived from the \texttt{assert} statement, we use the SMT solver to check if there is an assignment that satisfies the state yet violates the assertion predicate. If such assignment exists, we deem that assertion as invalid.

\subsection*{Example}
Given variables $x,y,z$ and the following state:
\begin{align*}
&C[x] = \top \quad C[y] = 11 \quad C[z] = \top \\
&D[x,y] = 2 \quad D[x,z] = -10 \quad D[y,z] = \top
\end{align*}

The state formula would be:
\begin{equation*}
state_\alpha = \overbrace{(y=11)}^{const_\alpha}\land\overbrace{(x-y=2)\land(x-z=-10)}^{diff_\alpha}
\end{equation*}
