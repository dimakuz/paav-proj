This example shows again that it can deduce lengths between nodes in different linked lists created in different ways.

Here we create two lists, one of length 3 (pointed to by \texttt{x}) and one of an arbitrary length (pointed to by \texttt{y}). Now we create two lists (pointed to by \texttt{z1} and \texttt{z2}). \texttt{z1} list is created by first going over \texttt{y} list and in each iteration going over \texttt{x} list in a nested loop, adding a new node to \texttt{z1}. \texttt{z2} list is created by first going over \texttt{x} list and in each iteration going over \texttt{y} list, adding a new node to \texttt{z2}. We can see in \texttt{example5-output/L701.png} that both lists are of length $3\cdot PL6 + 8$ in the end. 
