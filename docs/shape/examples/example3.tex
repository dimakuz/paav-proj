This example demonstrates that analysis is able to track sizes of nodes in different linked lists during iteration. First we create two singly linked lists pointed to by \texttt{x} and \texttt{y}, where the list pointed to by \texttt{y} is double the size of the list pointed to by \texttt{x}.

Now we iterate over the \texttt{x} and \texttt{y} lists simultaneously until \texttt{x} list end is reached, creating a new list pointed to by \texttt{z} on the way.

Then we iterate over \texttt{z} list to reach its end. We can assert at this point that \texttt{LEN y yy} is equals to \texttt{LEN z zz} (and also to \texttt{LEN x xx}). Note that \texttt{yy} is in the "middle" of \texttt{y} list, and until the end of this list we have exactly the same number of nodes as we have passed.

In the structures list, we find a structure where we see that \texttt{LEN z zz} is $1+(PL6+1)+1=PL6+3$, same as \texttt{LEN y yy}. Note that length from \texttt{yy} to the end of the list is also $PL6+3$.
