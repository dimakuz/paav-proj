\section*{Abstract Domain}
Our abstract state is defined as a 3-tuple of the from $(M,S,A)$. Where:

\begin{itemize}
	\item $M: Symbols \rightarrow 2^{\{O, E\}}$ - a mapping that tracks parity of symbols in the program. Each of the symbols is mapped to one of the values:
	\begin{itemize}
		\item $\emptyset=\bot$ - if no info about symbol parity is known.
		\item $\{O\}$ - if symbol is known to be odd-valued.
		\item $\{E\}$ - if symbol is known to be even-value.
		\item $\{O,E\}=\top$ - if symbol can be either odd or even.
	\end{itemize}
	\item $S: Symbols \rightarrow 2^{Symbols}$ - a mapping of symbols to symbol set of potentially similar parity.
	
	For example, given $x, y$, if we know that $x$ is now equal to $y$, after the assignment we'll change the state so that we'll have $y \in S[x]$.
	
	\item $A: Symbols \rightarrow 2^{Symbols}$ - a mapping of symbols to symbol set of potentially opposing parity (i.e. for two given symbols, one is odd and other is even).

\end{itemize}
Our abstract domain is then all possible assignments to the 3-tuple.
	
The join operation is defined as following: given $\alpha_1=(M_1,S_1,A_1)$ and $\alpha_2=(M_2,S_2,A_2)$ abstract states:
\begin{align*}
&\alpha_1\sqcup\alpha_2=(M_1\sqcup M_2, S_1 \sqcup S_2, A_1 \sqcup A_2) \\
\forall s \in Symbols&:  \\
&(M_1\sqcup M_2)[s] =M_1[s] \cup M_2[s] \\
&(S_1\sqcup S_2)[s] =S_1[s] \cup S_2[s] \\
&(A_1\sqcup A_2)[s] =A_1[s] \cup A_2[s]
\end{align*}

When verifying \texttt{assert PRED} statements, our analysis computes satisfiability of the following formula: $state_\alpha\land\neg pred$. We'll detail how $state_\alpha$ is obtained from $\alpha = (M,S,A)$:
\begin{align*}
state_\alpha & = basis_\alpha \land modulo_\alpha \land relations_\alpha \\
\\
basis_\alpha &= \bigwedge\{s_{even}\leftrightarrow\neg s_{odd} \mid s \in Symbols\}
\end{align*}
In this model, atoms are of the form $sym_{even}$ and $sym_{odd}$, meaning that $sym$ is even or odd respectively. We know no variable can be both odd and even at the same time, and cannot also be neither of them. $basis_\alpha$ formula restricts those pairs of atoms so exactly one can be true.
\begin{align*}
modulo_\alpha &= moduloE_\alpha \land moduloO_\alpha \\
moduloE_\alpha &= \bigwedge\{s_{even} \mid s \in Symbols \land M[s] = \{E\}\} \\
moduloO_\alpha &= \bigwedge\{s_{odd} \mid s \in Symbols \land M[s] = \{O\}\}
\end{align*}
$modulo_\alpha$ further restricts assignments to the atoms, if we know for sure that a variable is odd or even (through $M$ mapping), then we'll add clause that requires the relevant atom to be true.
\begin{align*}
relations_\alpha &= \bigwedge \{relationsE_{sym} \land relationsO_{sym}  \mid sym \in Symbols\} \\
relationsE_{sym} &= \left[
	\left[
		\bigwedge \{s_{even} \mid s \in S[sym]\}
	\right] \land  \left[
		\bigwedge \{s_{odd} \mid s \in A[sym]\}
	\right]
\right]
\rightarrow sym_{even} \\
relationsO_{sym} &= \left[
	\left[
		\bigwedge \{s_{odd} \mid s \in S[sym]\}
	\right] \land  \left[
		\bigwedge \{s_{even} \mid s \in A[sym]\}
	\right]
\right]
\rightarrow sym_{odd} \\
\end{align*}
$relations_\alpha$ encodes the relations between variables. If concrete parity information is unknown for variable $x$, we can deduce that it is even, if all variables in $S[x]$ are even and all variables in $A[x]$ are odd (and the other way around for odd case).

With the above formula, and the formula derived from the \texttt{assert} statement, we use the SAT solver to check if there is an assignment that satisfies the state yet violates the assertion predicate. If such assignment exists, we decide that assertion is invalid.

\subsection*{Example}
Given variables $x,y,z$ and the following state:
\begin{align*}
&M[x] = \{O\} \quad M[y] = \top \quad M[z] = \top \\
&S[x] = \emptyset \quad S[y] = \{x\} \quad S[z] = \emptyset \\
&A[x] = \emptyset \quad A[y] = \emptyset \quad A[z] = \{y\}
\end{align*}

The state formula would be:
\begin{align*}
state =& \overbrace{\parfBasis{x}\land\parfBasis{y}\land\parfBasis{z}}^{basis} \land \overbrace{x_{odd}}^{modulo}\land \\
&\overbrace{(x_{even}\rightarrow y_{odd}) \land (x_{odd} \rightarrow y_{odd}) \land 
            (y_{odd} \rightarrow z_{even}) \land (y_{even} \rightarrow z_{odd})}^{relations}
\end{align*}