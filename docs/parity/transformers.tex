\section*{Abstract Transformers}
We'll now define the transformers we used. All nodes are initialized with the $\alpha_\bot$ state, defined as:
\begin{align*}
	&\alpha_\bot=(M_\bot, S_\bot, A_\bot) \\
	\forall s \in Symbols&:
	M_\bot[s] =\emptyset  \quad
	S_\bot[s] =\emptyset \quad
	A_\bot[s] =\emptyset
\end{align*}
For any statement we'll denote the tagged mappings as mappings after the transformation:
\begin{equation*}
	\atrans{stmt}((M, S, A)) = (M^\prime,S^\prime,A^\prime)
\end{equation*}
For \texttt{skip} statement, the state is not modified:
\begin{equation*}
	M^\prime = M, \quad S^\prime = S, \quad A^\prime = A
\end{equation*}
For \texttt{i := ?} statement, all information regarding $i$ is removed. Both concrete parity info of the variable, and it's parity with regard to other variables:
\begin{align*}
M^\prime[x] = & \left.
	\begin{cases}
		M[x], & x\ne i \\
		\top, & x = i
	\end{cases}
\right\}\\
S^\prime[x] = & \left.
	\begin{cases}
		S[x] \setminus \{i\}, & x\ne i \\
		\emptyset, & x = i
	\end{cases}
\right\}\\
A^\prime[x] = & \left.
\begin{cases}
	A[x] \setminus \{i\}, & x\ne i \\
	\emptyset, & x = i
\end{cases}
\right\}
\end{align*}
For \texttt{i := K} statement, concrete parity of $i$ is stored, and it's relation to other variables is removed:
\begin{align*}
M^\prime[x] = & \left.
	\begin{cases}
		M[x], & x\ne i \\
		\{E\}, & x = i \wedge K \text{ is even} \\
		\{O\}, & x = i \wedge K \text{ is odd}
	\end{cases}
\right\}\\
S^\prime[x] = & \left.
	\begin{cases}
		S[x] \setminus \{i\}, & x\ne i \\
		\emptyset, & x = i
	\end{cases}
\right\}\\
A^\prime[x] = & \left.
\begin{cases}
	A[x] \setminus \{i\}, & x\ne i \\
	\emptyset, & x = i
\end{cases}
\right\}
\end{align*}
For \texttt{i := i} the state is not transformed.

For \texttt{i := j} statement, parity info is copied over from $j$ to $i$, and a relation that $i$ is of same parity as $j$ is added to the state:
\begin{align*}
M^\prime[x] = & \left.
	\begin{cases}
		M[x], & x\ne i \\
		M[j], & x = i
	\end{cases}
\right\}\\
S^\prime[x] = & \left.
	\begin{cases}
		S[x] \setminus \{i\}, & x\ne i \\
		\{j\}, & x = i
	\end{cases}
\right\}\\
A^\prime[x] = & \left.
\begin{cases}
	A[x] \setminus \{i\}, & x\ne i \\
	\emptyset, & x = i
\end{cases}
\right\}
\end{align*}
For \texttt{i := i + 1} and \texttt{i := i - 1} statements, concrete parity of $i$ is flipped if known, plus, $i$'s relations to other variables is reversed:
\begin{align*}
M^\prime[x] = & \left.
	\begin{cases}
		M[x], & x\ne i \\
		\{E, O\} \setminus M[i], & x = i \wedge M[i] \notin \{\bot, \top\} \\
		M[i], & \text{otherwise}
	\end{cases}
\right\}\\
S^\prime[x] = & \left.
	\begin{cases}
		S[x], & x\ne i \wedge i \notin A[x]\\
		S[x] \cup \{i\}, & x\ne i \wedge i \in A[x]\\
		A[x], & x = i
	\end{cases}
\right\}\\
A^\prime[x] = & \left.
	\begin{cases}
		A[x], & x\ne i \wedge i \notin S[x]\\
		A[x] \cup \{i\}, & x\ne i \wedge i \in S[x]\\
		S[i], & x = i
	\end{cases}
\right\}
\end{align*}


For \texttt{i := j + 1} and \texttt{i := j - 1} statements, state is transformed in a similar way to straight assignment, except the opposite parity is used.
\begin{align*}
M^\prime[x] = & \left.
	\begin{cases}
		M[x], & x\ne i \\
		\{E, O\} \setminus M[j], & x = i \wedge M[j] \in \{E, O\} \\
		M[i], & \text{otherwise}
	\end{cases}
\right\}\\
S^\prime[x] = & \left.
	\begin{cases}
		S[x] \setminus \{i\}, & x\ne i \\
		\emptyset, & x = i 
	\end{cases}
\right\}\\
A^\prime[x] = & \left.
\begin{cases}
	A[x] \setminus \{i\}, & x\ne i \\
	\{j\}, & x = i
\end{cases}
\right\}
\end{align*}
For \texttt{assume TRUE} statements the state is not altered, for \texttt{assume FALSE} statements, state is transformed into bottom state ($\sqcup$ - neutral) so chaotic iteration will treat it as a dead end.

For \texttt{assume i = j} statements, if $i$ and $j$ are known to be of opposite parities, statement is treated as \texttt{assume FALSE}, otherwise, state is transformed to create relation of similar parity between the variables, and both are set to most concrete parity known about either of them:
\begin{align*}
M^\prime[x] = & \left.
	\begin{cases}
		M[i]\cap M[j], & x \in \{i,j\} \\
		M[x], & \text{ otherwise}
	\end{cases}
\right\}\\
S^\prime[x] = & \left.
\begin{cases}
	S[x] \cup \{j\}, & x = i \\
	S[x] \cup \{i\}, & x = j \\
	S[x], & \text{ otherwise}
\end{cases}
\right\}\\
A^\prime[x] = & A
\end{align*}

For \texttt{assume i != j} and \texttt{assume i != K} statements state is not transformed.

For \texttt{assume i = K} statements, state is checked for conflict against parity of $i$. If there's a conflict, statement is treated as  \texttt{assume FALSE}, otherwise parity information is augmented with constant's parity:
\begin{align*}
M^\prime[x] = & \left.
	\begin{cases}
		\{E\}, & x = i \wedge K \equiv_{2} 0 \\
		\{O\}, & x = i \wedge K \equiv_{2} 1 \\
		M[x], & \text{ otherwise}
	\end{cases}
\right\}\\
S^\prime[x] = & S \qquad
A^\prime[x] =  A
\end{align*}


Finally, for \texttt{assert PRED} statements, transformer is similar to \texttt{skip}, state is not transformed.

In addition to the aforementioned transformations, after each such transformation and join operation, the abstract state is checked for conflicts and if those appear, they are eliminated. Specifically, if a symbol appears to be of same parity and opposite parity in regards to another variable at the same time, the relation between this pair of variables is removed from the state.
