\section*{Abstract Domain}
Our abstract state is defined as a 2-tuple of the from $(S,D)$. Where:
\begin{itemize}
	\item $S: 2^{Symbols} \rightarrow \mathbb{Z}\cup \{\bot, \top\}$ - a mapping that tracks constant values of sets of variables.
	\item $D: (Symbols \times Symbols )\rightarrow \mathbb{Z}\cup \{\bot, \top\}$ - a matrix that tracks difference between values of variables.
	
	For example, if $x$ is known to be greater than $y$ by 2, then $D[x,y] = 2$ and $D[y,x] = -2$.

\end{itemize}
Our abstract domain is then all possible assignments to the 2-tuple.

In order to control the complexity of analysis, implementation limits $S$ element. While algorithm assumes any element of $2^{Symbols}$, code only initializes $\bigcup_{1\le i < k} {Symbols}^i$, i.e. sets of up to $k$ elements.

We'll define $\sqcup$ and $\sqcap$ between elements of $\mathbb{Z}\cup \{\bot, \top\}$ as we have defined them for the lattice we seen in class, for example, $\sqcup$ is defined as:
\begin{itemize}
	\item $K \sqcup \bot = K$
	\item $K \sqcup \top = \top$
	\item $K \sqcup K = K$
	\item $K \sqcup K^\prime = \top$
	\item $\bot \sqcup \top = \top$
\end{itemize}
We'll also define the above $\sqcup$ commutative, meaning that $a\sqcup b \equiv b \sqcup a$.

The join operation is defined as following: given $\alpha_1=(S_1, D_1)$ and $\alpha_2=(S_2,D_2)$ abstract states:
\begin{align*}
&\alpha_1\sqcup\alpha_2=(S_1\sqcup S_2, D_1 \sqcup D_2) \\
\forall s \in Symbols&:  \\
&(S_1\sqcup S_2)[s] = \left.
	\begin{cases}
		S_1[s]\sqcup S_2[s], & s \in Symbols
	\end{cases}
\right\} \\
&(D_1\sqcup D_2)[s_1, s_2] = \left.
	\begin{cases}
		D_1[s_1, s_2]\sqcup D_2[s_1, s_2], & s \in Symbols
	\end{cases}
\right\}
\end{align*}

When verifying \texttt{assert PRED} statements, our analysis computes satisfiability of the following formula: $state_\alpha\land\neg pred$. We'll detail how $state_\alpha$ is obtained from $\alpha = (S,D)$:
\begin{align*}
state_\alpha & = sums_\alpha \land diff_\alpha \\
sums_\alpha &= \bigwedge\{s_1+\ldots+s_n = v \mid S[\{s_1,\ldots,s_n\}] = v \wedge v \in \mathbb{Z}\}
\end{align*}
In this model, atoms are variables of the program. $sums_\alpha$ is the set of clauses that restricts variables to hold the sums known in state.
\begin{equation*}
diff_\alpha = \bigwedge\{ x - y = D[x,y] \mid x,y\in Symbols \wedge D[x,y] \in \mathbb{Z}\}
\end{equation*}
$diff\alpha$ expresses the known differences between variables.

With the above formula, and the formula derived from the \texttt{assert} statement, we use the SMT solver to check if there is an assignment that satisfies the state yet violates the assertion predicate. If such assignment exists, we deem that assertion as invalid.

\subsection*{Example}
Given variables $x,y,z$ and the following state:
\begin{align*}
&S[\{x\}] = \top \quad S[\{y\}] = 15 \quad S[\{z\}] = \top \\
&S[\{x,z\}] = 30\quad S[\{x,y\}] = \top \quad S[\{y,z\}] = \top\\
&D[x,y] = -5 \quad D[x,z] = -10 \quad D[y,z] = \top
\end{align*}

The state formula would be:
\begin{equation*}
state_\alpha = \overbrace{(y=15)\wedge(x+z=30)}^{sums_\alpha}\land\overbrace{(x-y=-5)\land(x-z=-10)}^{diff_\alpha}
\end{equation*}
