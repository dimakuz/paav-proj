\section*{Abstract Transformers}

We'll now define the transformers we used. All nodes are initialized with the $\alpha_\bot$ state, defined as:
\begin{align*}
	&\alpha_\bot=(S_\bot, D_\bot) \\
	\forall s_1,s_2 \in Symbols&, s^\prime \in 2^{Symbols}:
	S_\bot[s^\prime] =\emptyset \quad
	D_\bot[s_1,s_2] =\emptyset
\end{align*}
For any statement we'll denote the tagged mappings as mappings after the transformation:
\begin{equation*}
	\atrans{stmt}((S,D)) = (S^\prime,D^\prime)
\end{equation*}
For \texttt{skip} statement, the state is not modified:
\begin{equation*}
	S^\prime = S, \quad D^\prime = D
\end{equation*}
For \texttt{i := ?} statement, all information regarding $i$ is removed. Both sums that include $i$ (with concrete values), and $i$'s difference to other variables:
\begin{align*}
S^\prime[x] = & \left.
	\begin{cases}
		S[x], & i \notin x \\
		\top, & i \in x
	\end{cases}
\right\}\\
D^\prime[s_1, s_2] = & \left.
	\begin{cases}
		D[s_1,s_2], & s_1 \ne i \wedge s_2 \ne i \\
		\top, & \text{otherwise}
	\end{cases}
\right\}
\end{align*}
For \texttt{i := K} statement, the transformer is split into 2 cases:
\begin{itemize}
\item If $i$'s previous value is known, i.e. $d = (K - S[\{i\}]) \in \mathbb{Z}$, then $d$ is used to augment the $i$'s differences and sums:
\begin{align*}
S^\prime[x] = & \left.
	\begin{cases}
		S[x], & i \notin x \\
		S[x] + d, & i \in x
	\end{cases}
\right\}\\
D^\prime[s_1, s_2] = & \left.
	\begin{cases}
		0, & s_1 = s_2 = i \\
		D[s_1,s_2] - d, & s_1 = i \\
		D[s_1,s_2], & \text{otherwise}
	\end{cases}
\right\}
\end{align*}
\item Otherwise, I'd value is unknown, therefore its $d$ is unavailable as well. In this case, we remove values of known sums that include $i$ and reset it's deltas from other variables.
\begin{align*}
S^\prime[x] = & \left.
	\begin{cases}
	    K, & i = \{i\} \\
   		\top, & i \in x \wedge |x| > 1 \\
		S[x], & i \notin x 
	\end{cases}
\right\}\\
D^\prime[s_1, s_2] = & \left.
	\begin{cases}
		0, & s_1 = s_2 = i \\
		\top, & s_1 = i \\
		D[s_1,s_2], & \text{otherwise}
	\end{cases}
\right\}
\end{align*}
\end{itemize}

For \texttt{i := i} the state is not transformed.

For \texttt{i := j} statement, we again, break transformation into 2 cases:

\begin{itemize}
\item If $i$'s old  and $j$'s current values are known, then we can use $d = (S[\{j\}] - S[\{i\}])$ to augment known sums and deltas. The transformation is:
\begin{align*}
S^\prime[x] = & \left.
	\begin{cases}
		S[x], & i \notin x \\
		S[x] + d, & i \in x
	\end{cases}
\right\}\\
D^\prime[s_1, s_2] = & \left.
	\begin{cases}
		0, & s_1 = i \wedge s_2 = j \\
		D[s_1,s_2] - d, & s_1 = i \\
		D[s_1,s_2], & \text{otherwise}
	\end{cases}
\right\}
\end{align*}
\item Otherwise, delta is unknown and we have to reset sums and deltas of $i$ (other than with $j$):
\begin{align*}
S^\prime[x] = & \left.
	\begin{cases}
		S[x], & i \notin x \\
		\top, & i \in x
	\end{cases}
\right\}\\
D^\prime[s_1, s_2] = & \left.
	\begin{cases}
		0, & s_1 = i \wedge s_2 = j \\
		\top, & s_1 = i \\
		D[s_1,s_2], & \text{otherwise}
	\end{cases}
\right\}
\end{align*}
\end{itemize}

For \texttt{i := i + 1}, sums involving $i$ are increased by 1 if known, and $i$'s differences from other variables are adjusted by 1 as well:
\begin{align*}
S^\prime[x] = & \left.
	\begin{cases}
		S[x], & i \notin x \\
		S[x] + 1, & i \in x
	\end{cases}
\right\}\\
D^\prime[s_1, s_2] = & \left.
	\begin{cases}
		D[s_1,s_2] - 1, & s_1 = i \\
		D[s_1,s_2], & \text{otherwise}
	\end{cases}
\right\}
\end{align*}
The transformer for \texttt{i := i - 1} is similar with opposite 1 signs .

For \texttt{i := j + 1} and \texttt{i := j - 1} statements we consider two cases, depending on knowledge of $S[\{j\}]$ and $S[\{i\}]$. The minus case is similar except for minor sign adjustments.
\begin{itemize}
\item If $d=S[\{j\}]-S[\{i\}]$ is a concrete value, then:
\begin{align*}
S^\prime[x] = & \left.
	\begin{cases}
		S[x], & i \notin x \\
		S[x] + d + 1, & i \in x
	\end{cases}
\right\}\\
D^\prime[s_1, s_2] = & \left.
	\begin{cases}
		1, & s_1 = i \wedge s_2 = j \\
		D[s_1,s_2] - (d + 1), & s_1 = i \\
		D[s_1,s_2], & \text{otherwise}
	\end{cases}
\right\}
\end{align*}
\item Otherwise, if $d$ is not a concrete value, we have to remove information involving $i$ form the state:
\begin{align*}
S^\prime[x] = & \left.
	\begin{cases}
		S[x], & i \notin x \\
		\top, & i \in x
	\end{cases}
\right\}\\
D^\prime[s_1, s_2] = & \left.
	\begin{cases}
		1, & s_1 = i \wedge s_2 = j \\
		\top, & s_1 = i \\
		D[s_1,s_2], & \text{otherwise}
	\end{cases}
\right\}
\end{align*}
\end{itemize}


For \texttt{assume TRUE} statements the state is not altered, for \texttt{assume FALSE} statements, state is transformed into bottom state ($\sqcup$ - neutral) so chaotic iteration will treat it as a dead end.

For \texttt{assume i = j} statements, if $i$ and $j$ are known to be of different value (either through constant propagation or their de ltas), statement is treated as \texttt{assume FALSE}, otherwise, state is transformed to include that delta between $i$ and $j$ is 0:
\begin{align*}
S^\prime = & S \\
D^\prime[s_1, s_2] = & \left.
	\begin{cases}
		0, & s_1 = i \wedge s_2 = j \\
		D[s_1,s_2], & \text{otherwise}
	\end{cases}
\right\}
\end{align*}

For \texttt{assume i != j} and \texttt{assume i != K} statements state is checked for conflicts, i.e. if variable $i$ is of value $K$ or equal ot $j$, then state is transformed to $\alpha_\bot$ and statement is treated as \texttt{assume FALSE}.

For \texttt{assume i = K} statements, state is checked for conflict with the already known value of $i$. If there's a conflict, statement is treated as \texttt{assume FALSE}, otherwise, constant value is stored in the state:
\begin{align*}
S^\prime[x] = & \left.
	\begin{cases}
		K, & x = \{i\} \\
		S[x], & \text{otherwise}
	\end{cases}
\right\}\\
D^\prime= & D
\end{align*}


Finally, for \texttt{assert PRED} statements, transformer is similar to \texttt{skip}, state is not transformed.

In addition to the aforementioned transformations, after each such transformation and join operation, the abstract state is augmented by deducing new deltas and sums.

\begin{itemize}
\item \textit{Deduction of deltas from singleton sums} - in transformers such as assignment of constant value to a variable, we learn new singleton sum value. When we do that, we immediately know its deltas from from all other singletons we know constant values of. We use this post-transform augmentation to discover new delta values:
\begin{align*}
S^\prime = & S \\
D^\prime[s_1, s_2] = & \left.
	\begin{cases}
		S[s_1] - S[s_2], & D[s_1, s_2] \notin \mathbb{Z} \land S[\{s_1\}],S[\{s_2\}] \in \mathbb{Z} \\
		D[s_1,s_2], & \text{otherwise}
	\end{cases}
\right\}
\end{align*}
\item \textit{Deduction of deltas from other known deltas} - for any 3 variables $a,b,c$ it holds that $(a-b) + (b-c) = a-c$. We can use that to deduce extra deltas from already known ones:
\begin{align*}
S^\prime = & S \\
D^\prime[s_1, s_2] = & \left.
	\begin{cases}
		D[s_1,s_3] + D[s_3, s_2], & D[s_1, s_2] \notin \mathbb{Z} \land \exists s_3: D[s_1,s_3],D[s_3,s_2] \in \mathbb{Z} \\
		D[s_1,s_2], & \text{otherwise}
	\end{cases}
\right\}
\end{align*}

\item \textit{Deduction of singletons from deltas} - given 2 variables, $a$ and $b$ with information that $a=K$ and $b-a=N$ we can deduce $b = K - N$. This post transformation tries to deduce the unknown variable values from known variable values and their deltas.
\begin{align*}
S^\prime[x] = & \left.
	\begin{cases}
		S[\{j\}] + D[i,j], & x=\{i\}\land S[x] \notin \mathbb{Z} \land \exists j: D[i,j],S[\{j\}]\in \mathbb{Z} \\
		S[x], & \text{otherwise}
	\end{cases}
\right\}\\
D^\prime= & D
\end{align*}

\item \textit{Deduction of sum combinations} - when new sum information appears, we can use it to provide/deduce new sums. For instance ${a,b}$ and ${c,d}$ can be used to deduce sum on ${a,b,c,d}$.
\begin{align*}
S^\prime[x] = & \left.
	\begin{cases}
		S[y] + S[z], & S[x] \notin \mathbb{Z} \land \exists y,z: S[y],S[z]\in \mathbb{Z}\land y\cup z = x \land y \cap z = \emptyset \\
		S[x], & \text{otherwise}
	\end{cases}
\right\}\\
D^\prime= & D
\end{align*}
\item \textit{Deduction of sub sums} - just like we can deduce value of $a$ from $b=N$ and $a-b=K$, we can deduce $a+b$ from $a+b+c+d+e=K$ and $c+d+e=N$. This transformation uses already known sums to deduce more sub sums.
\begin{align*}
S^\prime[x] = & \left.
	\begin{cases}
		S[z]- S[y], & S[x] \notin \mathbb{Z} \land \exists y,z: S[y],S[z]\in \mathbb{Z}\land x = z \setminus y \land y \cap x = \emptyset \\
		S[x], & \text{otherwise}
	\end{cases}
\right\}\\
D^\prime= & D
\end{align*}
\end{itemize}

The above transformations are applied in a loop until a fixed point is reached. Process is monotone because:
\begin{enumerate}
\item Transformers modify values of $S,D$ mappings.
\item $S,D$ mappings are finite and of constant size for the duration of analysis.
\item Only values of $\top,\bot$ are transformed, the result is always an integer.
\item Thus, each iteration, there's less and less values to operate on.
\end{enumerate}