\section*{Abstract Transformers}

We'll now define the transformers we used. All nodes are initialized with the $\alpha_\bot$ state, defined as:
\begin{align*}
	&\alpha_\bot=(C_\bot, D_\bot) \\
	\forall s,s_1,s_2 \in Symbols&:
	C_\bot[s] =\emptyset \quad
	D_\bot[s_1,s_2] =\emptyset
\end{align*}
For any statement we'll denote the tagged mappings as mappings after the transformation:
\begin{equation*}
	\atrans{stmt}((C,D)) = (C^\prime,D^\prime)
\end{equation*}
For \texttt{skip} statement, the state is not modified:
\begin{equation*}
	C^\prime = C, \quad D^\prime = D
\end{equation*}
For \texttt{i := ?} statement, all information regarding $i$ is removed. Both concrete value (if known), and it's difference to other variables:
\begin{align*}
C^\prime[x] = & \left.
	\begin{cases}
		C[x], & x\ne i \\
		\top, & x = i
	\end{cases}
\right\}\\
D^\prime[s_1, s_2] = & \left.
	\begin{cases}
		D[s_1,s_2], & s_1 \ne i \wedge s_2 \ne i \\
		\top, & \text{otherwise}
	\end{cases}
\right\}
\end{align*}
For \texttt{i := K} statement, concrete value of $i$ is stored, and it's relation to other variables is removed:
\begin{align*}
C^\prime[x] = & \left.
	\begin{cases}
		C[x], & x\ne i \\
		K, & x = i 
	\end{cases}
\right\}\\
D^\prime[s_1, s_2] = & \left.
	\begin{cases}
		D[s_1,s_2], & s_1 \ne i \wedge s_2 \ne i \\
		\top, & \text{otherwise}
	\end{cases}
\right\}
\end{align*}
For \texttt{i := i} the state is not transformed.

For \texttt{i := j} statement, value info is copied over from $j$ to $i$, differences between $i$ and other variables is erased, and difference between $i$ and $j$ is set to 0.
\begin{align*}
C^\prime[x] = & \left.
	\begin{cases}
		C[x], & x\ne i \\
		C[j], & x = i
	\end{cases}
\right\}\\
D^\prime[s_1,s_2] = & \left.
	\begin{cases}
		0, & \{s_1,s_2\} = \{i,j\} \\
		\top, & i \in \{s_1, s_2\} \\
		D[s_1,s_2], & otherwise
	\end{cases}
\right\}\\
\end{align*}
For \texttt{i := i + 1}, constant value of $i$ is increased by 1 if known, and $i$'s differences from other variables is adjusted by 1 as well:
\begin{align*}
C^\prime[x] = & \left.
	\begin{cases}
		C[x] + 1, & x = i \wedge C[i] \ne \top \\
		C[x], & \text{otherwise}
	\end{cases}
\right\}\\
D^\prime[s_1,s_2] = & \left.
	\begin{cases}
		D[s_1,s_2] + 1, & s_1 = i \wedge D[s_1,s_2] \in \mathbb{Z}\\
		D[s_1,s_2] - 1, & s_2 = i \wedge D[s_1,s_2] \in \mathbb{Z}\\
		D[s_1,s_2], & \text{otherwise}
	\end{cases}
\right\}
\end{align*}

For \texttt{i := j + 1} and \texttt{i := j - 1} statements, state is transformed in a similar way to straight assignment, except the opposite parity is used.
\begin{align*}
C^\prime[x] = & \left.
	\begin{cases}
		C[j] + 1, & x = i \wedge C[j] \in \mathbb{Z} \\
		C[x], & \text{otherwise}
	\end{cases}
\right\}\\
D^\prime[s_1,s_2] = & \left.
	\begin{cases}
		1, & s_1 = i \wedge s_2 = j\\
		\top, & s_1 = i \wedge s_2 \ne j\\
		D[s_1,s_2], & \text{otherwise}	
	\end{cases}
\right\}\\
\end{align*}

For \texttt{i := i - 1} and \texttt{i := j - 1} statements, abstract state is transformed into similar manner to increment case counterparts, but this time with inverted signs.

For \texttt{assume TRUE} statements the state is not altered, for \texttt{assume FALSE} statements, state is transformed into bottom state ($\sqcup$ - neutral) so chaotic iteration will treat it as a dead end.

\part*{FIXME complete below}
For \texttt{assume i = j} statements, if $i$ and $j$ are known to be of opposite parities, statement is treated as \texttt{assume FALSE}, otherwise, state is transformed to create relation of similar parity between the variables, and both are set to most concrete parity known about either of them:
\begin{align*}
M^\prime[x] = & \left.
	\begin{cases}
		M[i]\cap M[j], & x \in \{i,j\} \\
		M[x], & \text{ otherwise}
	\end{cases}
\right\}\\
S^\prime[x] = & \left.
\begin{cases}
	S[x] \cup \{j\}, & x = i \\
	S[x] \cup \{i\}, & x = j \\
	S[x], & \text{ otherwise}
\end{cases}
\right\}\\
A^\prime[x] = & A
\end{align*}

For \texttt{assume i != j} and \texttt{assume i != K} statements state is not transformed.

For \texttt{assume i = K} statements, state is checked for conflict against parity of $i$. If there's a conflict, statement is treated as  \texttt{assume FALSE}, otherwise parity information is augmented with constant's parity:
\begin{align*}
M^\prime[x] = & \left.
	\begin{cases}
		{E}, & x = i \wedge K \equiv_{2} 0 \\
		{O}, & x = i \wedge K \equiv_{2} 1 \\
		M[x], & \text{ otherwise}
	\end{cases}
\right\}\\
S^\prime[x] = & S \qquad
A^\prime[x] =  A
\end{align*}


Finally, for \texttt{assert PRED} statements, transformer is similar to \texttt{skip}, state is not transformed.

In addition to the aforementioned transformations, after each such transformation and join operation, the abstract state is checked for conflicts and if those appear, they are eliminated. Specifically, if a symbol appears to be of same parity and opposite parity in regards to another variable at the same time, the relation between this pair of variables is removed from the state.
